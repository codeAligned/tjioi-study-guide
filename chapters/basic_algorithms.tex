\chapter{Basic Algorithms}

Now that we've discussed some of the tools programmers use to write algorithms, let's discuss the actual algorithms!  Coders use many different techniques in order to solve problems, and many of the basic methods will be discussed in this section.

\section{Brute force}

A \textbf{brute force} algorithm is one that tries every possible solution to a problem in hopes of finding one that works. Though this may seem inefficient, it may be the best option in some problems where the number of possibilities is sufficiently small. Another way of looking at it is that instead of generating solutions directly, we generate a larger set of possible solutions and filter out the ones that work.

\begin{Problem}
Larry would like to visit $ N $ cities ($1 \leq N \leq 8$).  Given the distance between any two cities, determine the shortest possible path that Larry must take in order to visit every city exactly once.
\end{Problem}

For example, consider the above problem.\footnote{You might recognize this as the \textit{Traveling Salesman Problem}, which is a difficult problem to solve efficiently for large $ N $.  However, in this case, we are dealing with very small values of $ N $ to emphasize the use of the brute force technique.} As an initial approach to this problem, we would generate each different permutation of the $ N $ cities (such as 1, 2, 4, 5, 6, 7, 8, 3) and then determine the length of the path if we were to travel to those cities in that order, keeping track of the minimum path length.  The number of ways to choose the first city to visit is given by $ N $, the second by $ N - 1 $, the third by $ N - 2 $, and so on, and so the total number of ways to order $ N $ cities is given by $ N! $, the factorial of $ N $.  The largest value of $ N $ possible is $ N = 8 $, and so the most number of cases to check would be $ 8! = 40320 $.  This number is well less than $ 10^8 $, the number of operations per second that a computer is capable of performing, and so using a brute force solution to this problem is sufficient.

Note that in many cases, a brute force solution will \textbf{not} be fast enough, prompting us to search for more efficient algorithms.  However, we can still first attempt a problem using a brute force approach, and perform an analysis on whether or not it is sufficient to solve the problem within the time limit. If it is not, a brute force approach may still be a starting point for further refinement. For example, coding a brute force solution may help us reveal repeated calculations, which we can take advantage of using \textbf{dynamic programming}, discussed later in section~\ref{sec:dp}.


\section{Greedy algorithms} \label{greedy}

A \textbf{greedy algorithm} describes an algorithm that, in order to find the best solution, at each step, chooses the option that appears to best lead it in the right direction.  The assumption made by the Greedy Algorithm is that if, at each intermediate step, we make the best possible choice at that step, we will arrive at the best overall result.

This seems like a fairly reasonable strategy, as by following the best choice at each step, we should be led in the right direction.  For example, consider this problem:

\begin{Problem}
Kevin has $ \$N $ ($ 1 \leq N \leq 1000 $) and would like to make change with the least number of bills possible.  If there are bills worth $ \$1 $, $ \$5 $, $ \$10 $, $ \$20 $, and $ \$100 $, determine the least number of bills needed to make exact change.
\end{Problem}

The optimal strategy here would be to first take as many $ 100 $-dollar bills as possible, then $ 20 $-dollar bills, then $ 10 $-dollar bills, and so on.  We can see this is true as each increasing bill amount is a multiple of the previous amount.  Therefore, if we have enough bills of a denomination to "trade in" for one bill of the next denomination, it will always reduce the total number of bills.  Suppose we begin with $ N $ bills of $ \$1 $.  We will trade in as many as possible for the largest denomination possible.  Using the remaining bills, we will trade in as many as possible for the next largest denomination.  We will repeat this process until we can no longer do so, and we have arrived at our solution.

Note that this method only works when increasing bill denominations are multiples of the previous denomination.  For example, if one only had bills of $ \$1 $, $ \$3 $, and $ \$4 $, and one wished to get change for $ \$6 $, the optimal solution would be $ 2 $ bills of $ \$3 $, whereas the greedy solution would call for one bill of $ \$4 $ and two bills of $ \$1 $. We'll discuss a variation of this problem in section~\ref{sec:knapsack}.

As demonstrated in that example, the greedy solution will not always produce the right answer.  One must think carefully to determine if the problem they are facing is one that can be solved using a greedy strategy.  The major drawback of the greedy algorithm is its short-sightedness. Imagine a chess player employing the greedy algorithm: the player will never sacrifice a piece, or move a piece backwards, as it makes the choice that is best specifically in that turn, and neither of those events are beneficial when viewed in the context of only that one turn.  In a complicated problem such as chess, taking a step backwards may result in a better end result.

In practice, the greedy algorithm is often used to find approximate solutions to a difficult problem.  By taking a greedy approach, we may be able to find an answer that is close to the actual answer, in order to provide a starting point for further optimization.

\section{Recursion} \label{recursion}

Recursion is a strategy that we can use in order to help us solve complicated problems. The idea of recursion involves solving a larger problem by breaking it down into smaller, self-similar problems. More generally, a \textit{recursive function} is a function that calls itself. We can apply this technique in a number of different ways, two of which we'll illustrate in this section.

\subsection{Recursive functions}

Recursion is most useful when the problem one is trying to solve depends on subproblems that are equivalent to the original.  For example:

\begin{Problem}
Given $ N $ ($ 1 \leq N \leq 100 $), determine the $ N $th Fibonacci number, where the $N$th Fibonacci number is defined as the sum of the $(N-1)$th and $(N-2)$th Fibonacci number. 
\end{Problem}

We can employ the use of a recursive solution in order to solve this problem, following immediately from the mathematical definition of Fibonacci numbers.

\begin{algorithm}[H]
\caption{Fibonacci}
\begin{algorithmic}

\Function{Fibonacci}{$n$}
    \If{$n = 1$ or $n = 2$}
        \State \Return 1
    \Else
        \State \Return \Call{Fibonacci}{$n-1$} + \Call{Fibonacci}{$n-2$}
    \EndIf
\EndFunction

\end{algorithmic}
\end{algorithm}

All recursive solutions must have two parts: the base case and the recursion.  The base case specifics where to stop --- without a base case, the recursion will go on until we hit a recursion limit!  In this solution, the base case is when $ N = 1 $ or $ N = 2 $.  The recursive part is the bulk of the solution.  Each call to the function calls the function again with a smaller argument, until it eventually becomes $ 1 $ or $ 2 $.  Because the Fibonacci problem can be broken down into smaller, self-similar pieces, recursion is a convenient way to solve this problem, though not a particularly efficient one. We'll discuss how to solve it with dynamic programming in section~\ref{sec:dp}.


\subsection{Recursive search} \label{recursive-search}
% Maybe a tree diagram for a visual?
We can also use recursion to generate and explore states. If from any given state in a problem, we can reach other states, then we can simply recur on the states that we generated to explore all possible states. This is known as a \textbf{depth-first search}.

We can represent the states we traverse with a search tree (we discussed trees in section~\ref{sec:trees}). The root node of the tree is the state from which we start. Then, its children are the other states that can be generated from it, and so on. Here's an example to demonstrate how we can use this idea to solve problems.

\begin{Problem}
Given an integer $n$, print out all possible permutations of the integers $1..n$. For example, if $n=3$, the permutations would be (1, 2, 3), (1, 3, 2), (2, 1, 3), (2, 3, 1), (3, 1, 2), (3, 2, 1).
\end{Problem}

\noindent This is easy to solve for a fixed value of $n$: for example, for $n=3$, we can just write $n$ for-loops. But given an arbitrary $n$, we can't just write $n$ for-loops. Instead, we can use a recursive strategy. We'll start by building each permutation of length 1. We can then recur on those results to build permutations of length 2, and so on.

% TODO: insert image for search tree

\begin{algorithm}[H]
\caption{Generating permutations}
\begin{algorithmic}

\State $used \gets \{false, false, \ldots, false\}$
\Function{GeneratePermutations}{depth, prefix}
    \If{$depth = n$}
        \State \Call{Print}{prefix}
        \State \Return
    \EndIf
    \For{$i = 1..n$}
        \If{not $used[i]$}
            \State $used[i] \gets true$
            \State \Call{GeneratePermutations}{$depth + 1$, $prefix \oplus i$}
            \State $used[i] \gets false$
        \EndIf
    \EndFor
\EndFunction

\end{algorithmic}
\end{algorithm}


\section{Searches}

In this section, searching refers specifically to searching for an item within an array. We want to know whether the item exists, and if so, we want to know its position. We will discuss two kinds of searches, each with their own advantages and disadvantages.

\subsection{Linear search}

The linear search is the most straightforward way of searching an array.  We iterate over each element of the array, and check to see if it is the element that we want. If there is a match, then the value is returned. If there is not, it will continue to search to the end of the data collection. As in the worst case, we will have to check every element of the array, the time complexity of linear search is $ O(n) $.  Although linear search is not the most efficient method, it will always work as each element is checked.

\subsection{Binary search}

Binary search is a more efficient method of searching.  Consider how one would find a word in a dictionary, for example, "cat".  We would flip to the middle of the dictionary, and examine words beginning with "m".  Because "cat" comes alphabetically before "m", we do not even have to search anywhere after "m", and can recursively repeat the process on the half of the dictionary that comes before "m".  Note that this relies on a key assumption: the data must already be sorted.  We can use the information that sorted data provides us to eliminate half the data at each step.  The number of operations that binary search needs is equivalent to how many times we can cut the data in half, until there is only one object left.  This is equivalent to the logarithm in base two of the size of the data, so the runtime of binary search is $ O(\log n) $.


\section{Sorts} \label{sec:sorts}

Sorting is a fundamental problem in computer science: one that is easy to understand and solve, but not as easy to solve \textit{efficiently}. Formally, the objective of a sorting algorithm is to rearrange the items in an array so that they are arranged in a well-defined order. Those items can be anything: numbers, strings, and even custom data types, so long as we have some way of determining how to order them.

% reword
In order to sort an array, we must first define how we intend to compare two objects in the array.  It doesn't make any sense to ask if Apple or Orange is greater until we state what we mean by "greater".  Different languages implement this differently: for example, in Java, we can determine the order in which to sort elements using the \texttt{Comparable} or \texttt{Comparator} interfaces.  These interfaces force us to write a method which compares an object with another, so that any object in the array must be "greater than", "less than", or "equal to" another object, which we define within the method.  The sorting algorithms we analyze in this section will be similar in that they can be performed using only comparison operations; thus, we refer to them as \textit{comparison-based sorts}.


\subsection{Selection sort}

The first algorithm we will consider is quite simple. First, we find the smallest element in an array; then, we put it into the first position. Next, we find the smallest element among the remaining elements, and put it into the second position. We continue until we have sorted the entire array.

We need some way of quantifying the performance of this algorithm, so we will count the number of compares and exchanges performed. We make $n-1$ total passes through the array, each of which requires one exchange. However, we make $N-i-1$ compares for $i \in [0, n-1]$, so the number of compares is approximately $N^2/2$. Note that the number of operations remains constant, even if the input data is already sorted!  The selection sort algorithm is $ O(n^2) $.


\subsection{Insertion sort}

Next, we consider another algorithm that is closer to how you might sort in real life. We loop through each of the items in an array, inserting each of them into the correct position among the items before it that have already been sorted. Coding insertion sort does require a bit of care.

If the array is already sorted, we only need to perform $n-1$ comparisons and $0$ exchanges total! However, in the worst case, we may need to perform $i$ comparisons and exchanges for $i \in [0, n-1]$. In total, we will perform approximately $n^2/2$ compares and exchanges.  Insertion sort is also $ O(n^2) $.


\subsection{Other sorts}

It turns out that there are sorts in only $O(n \log n)$ time. This time complexity is actually the lowest possible bound for comparison-based sorts! Among these sorts are \textit{merge sort}, \textit{quicksort}, and \textit{heapsort}, which you may have heard of, though we won't cover the details of their implementations here.
